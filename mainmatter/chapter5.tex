\chapter[Chapter 5: CITA on CHIP]{CITA on CHIP}

\section{Background}

The CITA on CHIP

Personalization is important
Analytics and tracking progress
It might be able to better serve women and minorities in STEM
Focus on student’s questions (it promotes individual thinking)
Track progress in an individualized way
Coach students through problems
Consider multiple iterations of the same lesson
Stress problem solving strategies
This program could be an equalizer in the classroom
This program might boost student confidence
This program might be good for international students with poor English
Stress multiple learning mechanisms
Break up each problem into tiny pieces
Provide immediate feedback
This program may help customize the curriculum
It is a non-linear way to solve problems

\section{Development}

We only continue exploring a topic if we feel comfortable in our environment and we feel it is worth it. We are economic creatures.

Advantages of Using CHIP
It has a huge database of problems from multiple textbooks
It already has the ability to collect statistics (underutilized)
It has a specialized instructor GUI
It has huge archives of past student data
There is no cost to the students
There is a lot of past experience using the site (faculty and IT)
There is a dedicated question and answer system
A dedicated staff handles bugs
It is on location in the physics building
It has no ties to other companies, businesses, buildings, people, etc.
It is self-contained

Disadvantages of Using CHIP
It has an out of date look
Hints are underutilized
There are no interactive graphics
We need more graphs related to learning outcomes
Uploading stuff is a pain
It has no ties to other companies, businesses, buildings, people, etc.

Goals
With the increased popularity of the online course as well as increased enrollment in the course as a whole, students need a homework website with an interactive learning tool that guides them through the homework no matter what time of day (i.e. different time zones, work schedules).  We want to design an computerized interactive TA (iTA) that is available at any time.

We want to design interactive components for each individual problem that can help students develop an approach to problem solving.  The CITA will guide students through problems in such a way as to develop critical thinking and problem solving skills.

High school teachers want to take the course to become accredited.  They see how physics should be thought out / structured.

High Priority Goals
- Update all content with CITA hints
- Update the statistical analysis parts of the site for student learning outcomes (interactive graphs of statistics)
- Review the site and bring all useful components to the forefront

Medium Priority Goals
- Update site graphics
-> Faculty First
-> Students Second
- Update figures (interactive?)

Low Priority Goals
- Blackboard/Canvas/Whatever Integration
- Mobile Support?

\subsection{Shallow CITA}
\subsubsection{Postscripts}
\subsubsection{Estimation Problems}

\subsection{Immersive CITA}
\subsubsection{None}
\subsubsection{Linear}
\subsubsection{Branching}

\section{What CITA Does Accomplish}

\section{What CITA Does Not Accomplish}

It is important to set boundaries.

\subsection{It Does Not Replace Teachers}

CITA in no way is designed to replace the job of a teacher in the classroom.

\subsection{It is Not a Course Management Tool}