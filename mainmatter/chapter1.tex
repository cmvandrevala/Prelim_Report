\chapter[Chapter 1: Introduction]{Introduction}

In this section, you should introduce your reader to your topic and purpose of your study. You should strive to convince your reader that the topic you have chosen to explore is societally significant and that there is a need to explore the topic. Thus, the writing here is persuasive. I will look for a well-defined and clearly articulated problem or issue that motivates your investigation. In addition, I should know the purpose of the paper very early in the section. Generally, well- written arguments will cite relevant literature that supports the need for investigation. By the end of this section, your reader should clearly know what you are going to study and be convinced that it is important to study in more depth. A clear purpose statement should be included that naturally leads to your review of the research literature.

\section{Background}

\section{Overview of PHYS 24100 at Purdue}

\subsection{Background}

PHYS 24100 is the introductory electricity and magnetism course for engineering majors at Purdue University. It is generally taken after students complete PHYS 17200 - the introductory mechanics course. The mechanics course is currently taught using the Matter and Interactions[CITATION NEEDED] textbook by Sherwood and Chabay. PHYS 24100 is

\subsection{Strengths}

One advantage of performing PER at Purdue University is the fact that the introductory physics classes
\subsection{Weaknesses}

\section{Overview of CHIP at Purdue}

\subsection{Strengths}

The CHIP program at Purdue is a well established system that has over a decade of in-the-field use. It has no ties to any external companies or organizations. The faculty at Purdue is free to update or modify the source code as they wish. Additionally, all of the CHIP servers are maintained on Purdue University campus in the physics building by our own IT support team.

CHIP has a dedicated staff of experts that run and maintain the system. This faculty works for Purdue University, so there is no conflict of interest outsourcing this job to other companies.

CHIP has a well developed

Finally, CHIP already has the framework in place to collect statistics on student performance during the semester. On a basic level, it is capable of recording student scores in a variety of assignments. However, it can also perform a basic statistical.

It has a huge database of problems from multiple textbooks
It already has the ability to collect statistics (underutilized)
It has a specialized instructor GUI
It has huge archives of past student data
There is no cost to the students
There is a lot of past experience using the site (faculty and IT)
There is a dedicated question and answer system
A dedicated staff handles bugs
It is self-contained

\subsection{Weaknesses}

Although CHIP has

CHIP has a lot of underutilized items. Statistical analysis is available on the system, but it is rarely used. Additionally, interactive tutorials have been implemented on 29 out of 139 (20.9\%) homework problems.

It has an out of date look
Hints are underutilized
There are no interactive graphics
We need more graphs related to learning outcomes
Uploading stuff is a pain

Finally, It has no ties to other companies, businesses, buildings, people, etc.

\section{Problem Statement}

