\chapter[Chapter 1: Introduction]{Introduction}

In this section, you should introduce your reader to your topic and purpose of your study. You should strive to convince your reader that the topic you have chosen to explore is societally significant and that there is a need to explore the topic. Thus, the writing here is persuasive. I will look for a well-defined and clearly articulated problem or issue that motivates your investigation. In addition, I should know the purpose of the paper very early in the section. Generally, well- written arguments will cite relevant literature that supports the need for investigation. By the end of this section, your reader should clearly know what you are going to study and be convinced that it is important to study in more depth. A clear purpose statement should be included that naturally leads to your review of the research literature.

\section{Background}

With the increased popularity of online classes, as well as a rise in enrollment in foundational courses for engineering, students need an interactive learning tool that guides them through homework problems when assistance from a professor, teaching assistant, or another student is unavailable. This is especially important in an environment where students with diverse schedules must attend class (i.e. managing work and class, teaching students abroad, etc.). Increased retention and student success in any course requires good communication and interaction among students and instructors.  However, the expectation of access to instructors at any time of the day, any day of the week, is unrealistic with the constant constriction in teaching staff because of reductions in funding.

\section{Overview of PHYS 24100 at Purdue}

\subsection{Background}

PHYS 24100 is the introductory electricity and magnetism course for engineering majors at Purdue University. It is generally taken after students complete PHYS 17200 - the introductory mechanics course. The mechanics course is currently taught using the Matter and Interactions[CITATION NEEDED] textbook by Sherwood and Chabay. PHYS 24100 is

\subsection{Strengths}

One advantage of performing PER at Purdue University is the fact that the introductory physics classes

\subsection{Weaknesses}

\section{Overview of CHIP at Purdue}

\subsection{Strengths}

The CHIP program at Purdue is a well established system that has over a decade of in-the-field use. One of its biggest advantages is that it has no ties to any external companies or organizations. The faculty at Purdue is free to update or modify the source code as they wish. All of the CHIP servers are maintained on Purdue University campus in the physics building by our own IT support team.

CHIP has a dedicated staff of experts that run and maintain the system. This faculty works for Purdue University, so there is no conflict of interest outsourcing this job to other companies.

CHIP has a well developed database of problems from a variety of textbooks. We have permission to use. Additionally, CHIP keeps a secure archive of scores from previous semesters.

CHIP already has the framework in place to collect statistics on student performance during the semester. On a basic level, it is capable of recording student scores in a variety of assignments. However, it can also perform a basic statistical.

CHIP offers a few advantages to students as well as faculty. CHIP comes at no extra cost to students - students only need to buy the course textbook rather than a course textbook and site license. CHIP has a mature gradebook that students can use to get feedback on assignments; if they are enrolled in an online course, this feedback is instantaneous because the entire system is self-contained. Additionally, when students ask questions or send error reports, they are talking directly with a course instructor rather than a secondary source. This facilitates communication and solves problems in a more timely fashion.

\subsection{Weaknesses}

Although CHIP has a successful history at Purdue, it is not without its problems. The CHIP program was coded in the late 90s and early 2000s. Thus, while the graphical user interface (GUI) was revolutionary at one time, it has begun to show its age. The overall look and layout of CHIP has not kept up with modern devices. Specifically, CHIP buttons and menus do not render well on small screens like cell phones, tablets, and netbooks.

CHIP has a lot of underutilized features. For example, different statistical analysis tools are available on the system, but they are rarely used due to the fact that they are either hard to find or require a learning curve.

Interactive tutorials have been implemented on only 29 out of 139 homework problems (about 20.9\% of the problems). These tutorials are certainly useful for solving the specific problem at hand, but they do little to inspire creativity and discussion in the classroom.

\section{Purpose Statement}

We want to develop an online ``Computerized Interactive Teaching Assistant'' (CITA) for the second semester of introductory physics (Electricity and Optics). This tutorial would be available to guide students through problem solving at any time. The most often heard statement to our teaching assistants for PHYS 241D and 241 from students attempting the homework or practice exams is, ``I don’t even know where to begin.''  With CITA we propose to design an interactive component for each individual problem that will guide the student through a focused strategy for problem solving.  CITA will guide students through problems in such a way as to develop critical thinking and problem solving skills with a thorough understanding of the physical concepts.

\section{Problem Statement}

