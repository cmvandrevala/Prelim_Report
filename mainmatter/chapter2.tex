\chapter[Chapter 2: Literature Review]{}

\section{Background}

Quasi-experiment
Bonferroni-Holm correction
Effect size
Cohen’s D
Grounded theory

\section{Paradigms of Learning}

\subsection{Behaviorist Theories}
\subsection{Cognitivist Theories}
\subsection{Constructivist, Social, and Situational Theories}
\subsection{Motivational and Humanist Theories}
\subsection{Design Theories and Models}
\subsection{Descriptive and Meta Theories}
\subsection{Identity Theotries}
\subsection{Miscellaneous Learning Theories and Models}

\section{My Focused Learning Theory}


http://www.learning-theories.com/

When learning new material students bring with them the prior knowledge that they had of the subject at hand. This knowledge includes both correct models of how the world works as well as incorrect, pre-existing notions of the universe. These pre-existing notions diverge from what the teacher is trying to teach.

Teaching learning gap (Redish and Steinberg 1999)

It is tempting to quantify how ingrained a model is in student's heads. Zeilik and Bisard call deeply ingrained misconceptions that are hard to change structural misconceptions. Alternatively, they call easily changed misconceptions factual misconceptions\cite{zeilik2000}. However, a pre- and post concept inventory showed no difference in gains between these two distinctions. However, even though the data showed no difference, they still found it useful to track misconceptions in order to develop lessons for their courses.

Zeilik et al 1997

I use the advice of Zeilik and  Bisard to track misconceptions\cite{zeilik2000}
If I want to teach as efficiently as possible, I need to target misconceptions.

Expert-Novice Theories
Epistemological Theories
Contemporary Learning Theories
Conceptual Change Theories
Generative vs. Revisionary Theories

Your research might indirectly touch on what is a good prompt when asking questions

\section{Interactive Engagement Methods}

\section{Background}

Interactive engagement (IE) methods were created to tackle the challenges of diverse classrooms. Students respond to a given lesson in a myriad of ways depending on their circumstances; today's professors have to deal with ``non-traditional students'' with a variety of backgrounds including older age students, full-time workers, part-time workers, commuters, military cadets, and special-needs students. Additionally, students have different educational backgrounds, career interests, and life goals that cannot be covered by a ``one size fits all'' approach\cite{novak1999}.

IE methods challenge students to think about concepts in a deeper, more meaningful way. They frown upon generic problems that can be solved in one or two steps - so called ``plug and chug'' problems. Instead, IE methods encourage students to frequently interact with their instructor, their teaching assistants, and their colleagues in order to solve deep, meaningful problems that draw together many areas of physics and mathematics. These problems often have real-world context that draw students interestCITATION.

Hake defines IE methods as those designed to promote conceptual understanding through ``heads-on'' and ``hands-on'' activities that yield immediate feedback through discussion. Conversely, traditional courses make little to no use of IE methods, relying primarily on passive lectures, ``cookbook'' labs, and ``plug and chug'' problemsCITATION.

\section{Examples}

\subsection{Just in Time Teaching}

\index{Just-in-Time Teaching}Just-in-Time Teaching (JiTT) is an IE method that is based on creating an ongoing conversation between the students learning the material and professors teaching the material. Professors assign readings and practice problems that students are required to answer electronically before class starts. Then, the professor reads these responses and customizes the material covered in class that day\cite{novak1999}.

The ``just in time'' strategy has seen success not only in the classroom, but also in the business world. By constantly assessing the state of the class, it allows for focused lessons that cover the topics that students are having the most trouble with. Additionally, students become more interested in the course material when they have an active hand in shaping the structure of the lessons.

\subsection{Activity Based Physics and Workshop Physics}

% Activity Based Physics is a multi-institutional project to sustain and enhance current efforts to render introductory physics courses more effective and exciting at both the secondary and college level.
%
% This program represents a collaborative effort by an informally constituted team of educational reformers to use the outcomes of physics education research along with flexible computer tools to develop activity-based models of physics instruction.
%
% This project includes the refinement of existing written materials, apparatus, instructional techniques, and computer software and hardware; the creation of new instructional materials and approaches; as well as classroom testing in different settings and dissemination.
%
% The refinement and development of instructional strategies and materials is informed by a comprehensive program of educational research.
%
% Workshop Physics takes
%
% is a new method of teaching calculus-based introductory physics without formal lectures. Instead students learn collaboratively through activities and observations. Observations are enhanced with computer tools for the collection, graphical display, analysis and modeling of real data.
%
% Workshop Physics classes consist of three, two-hour long sessions each week.
%
% \subsection{Tutorial-Based Instruction}

\subsection{Socratic Dialogue Inducing Methods}

Socratic Dialogue Inducing (SDI) methods teach students physics through hands-on experiements and discussion. These methods are designed to build student's mental model of a physics phenomenon by having students construct mental models of how they think the world works, testing these models with appropriate experiments, and then discussing their results in order to refine the model. Throughout the semester, the experiments become more and more sophisticated, refining their mental models of physics phenomena. Hake describes these labs as ``guided construction'' of physics concepts rather than ``guided discovery''\cite{hake1992}.

% \subsection{Active Learning Problem Sets}
% \subsection{Context-Rich Problem Based Learning}
%
% \section{Advantages of IE Methods}
%
% \subsection{Classroom Enjoyment}
%
% \subsection{Retention}
%
% IE methods have been shown to maximize the efficacy of the classroom session, where human instructors are present and to structure the out-of-class time for maximum learning benefit.
% Hake's study of IE classrooms across the United States has shown that students do, in fact, learn more from hands-on engagement than from any sort of passive learningCITATION.
%
% \subsection{Fostering Teamwork}
%
% Students that work with their colleagues and professors are learning not only the material in class, but also how to work on a team. This translates to a maximal retention of knowledge for the most students\cite{novak1999}.
%
\section{Disadvantages of IE Methods}

Although IE methods are a boon for the physics classroom, they do not come without their challenges. However, much time and effort has been put into developing strategies and materials to solve some of these challenges.

% \subsection{Time}
%
% \subsection{Cost}
%
% Cost can be prohibitive to set up interactive engagement methods. However, once the

% \subsection{Quantity of Material}
%
\subsection{Expectations Violation}

Students who are used to traditional teaching methods might feel overwhelmed by the new method of teaching. This is called expectations violation. It is a natural part of any classroom.

There are many ways to overcome expectations violation. First, students need to have a very clear understanding of what is expected of them. They need to know what should be turned in for credit, and what is only there for discussion and further learning. Additionally, students need to understand the philosophy behind the teaching style. For example, instructors need to tell students that going through a problem in multiple ways can help them tie concepts together that traditional methods would not be able to do.

\subsection{Training}

One of the main advantages of passive lecturing is that anybody can do it. As long as the professor has knowledge of the subject area, it is pretty easy to throw together a lesson in a short amount of time (whether or not that lesson is good is another story entirely). Conversely, IE methods take time and energy to implement. Being a skilled teacher involves collecting student feedback and changing how your lesson is structured accordingly. Professors need to take the time to effectively implement IE strategies in the classroom.

Additionally, if a classroom has used traditional teaching techniques for a long time, there may be a significant time commitment needed to build the new IE curriculum. This might be a serious commitment - especially in larger universities.

There are many software tools available that help alleviate these problems.

\subsection{Personal Feedback for Off-Campus Students}

One of the key aspects of IE methods is ongoing, personal feedback. This feedback can come in many forms - actual conversations with teachers and colleagues, online communication through email, interactive textbooks, and helpful feedback on assignments and grades, just to name a few.

Distance learning students might be at a disadvantage with IE methods because they physically cannot participate in the conversation that is occuring in class. These students must also deal with many other problems including time zone differences and lack of an internet connection. They must rely on electronic communication, creating a barrier between them and the class.

There are solutions to this problem. Programs like Adobe Connect, Cisco WebEx, and Skype make it simple for students to connect with the class as long as they have an internet connection. Additionally, courses like JiTT use asynchronous communciation methods - students do not need to submit their answers at the same time. Instead, students are free to submit their work anytime they want before a deadline, thus giving them the freedom to customize their schedule as needed.


