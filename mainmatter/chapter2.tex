\chapter[Chapter 2: Literature Review]{Literature Review}

\section{Constructivism}

When learning new material students bring with them the prior knowledge that they had of the subject at hand. This knowledge includes both correct models of how the world works as well as incorrect, pre-existing notions of the universe. These pre-existing notions might diverge from what the teacher is trying to teach.

Our model of learning in this study is constructivism. Constructivism is a learning philosophy grounded on the theory that we construct a personal understanding of the world we live in by reflecting on our experiences. We each generate our own mental models to make sense of our experiences. Learning is the process of adjusting our mental models with each new experience.

In the realm of physics, we create mental models of how the world works as we grow up. Some of these models are generally correct (i.e. if the floor is slippery, I might have trouble walking since I will slip). Other models are flawed (i.e. gravity always pulls things “downwards”). We will use a scaffolding approach to teach introductory physics to college level students.

It is tempting to quantify how ingrained a model is in student's heads. Zeilik and Bisard call deeply ingrained misconceptions that are hard to change structural misconceptions. Alternatively, they call easily changed misconceptions factual misconceptions\cite{zeilik2000}. However, a pre- and post concept inventory showed no difference in gains between these two distinctions. However, even though the data showed no difference, they still found it useful to track misconceptions in order to develop lessons for their courses.

\section{Scaffolding and Learning}

Scaffolding refers to a variety of techniques that are used to slowly move a student towards a firm understanding of the material. Teachers provide successive levels of temporary support that students can use to structure their learning. These levels of support are removed when the student no longer needs them.

In our case the scaffolding will come from the interactive examples. Our scaffolding is unique in that students will have multiple paths to follow on their tutorials, thus making the scaffolding more immersive. Students will have a chance to reflect on the examples and talk amongst each other through Piazza, keeping with the spirit of \gls{ie}.

\section{Interactive Engagement Methods}

\gls{ie} methods were created to tackle the challenges of diverse classrooms. Students respond to a given lesson in a myriad of ways depending on their circumstances; today's professors have to deal with ``non-traditional students'' with a variety of backgrounds including older age students, full-time workers, part-time workers, commuters, military cadets, and special-needs students. Additionally, students have different educational backgrounds, career interests, and life goals that cannot be covered by a ``one size fits all'' approach\cite{novak1999}.

\gls{ie} methods challenge students to think about concepts in a deeper, more meaningful way. They frown upon generic problems that can be solved in one or two steps - so called ``plug and chug'' problems. Instead, \gls{ie} methods encourage students to frequently interact with their instructor, their teaching assistants, and their colleagues in order to solve deep, meaningful problems that draw together many areas of physics and mathematics. These problems often have real-world context that draw students interest.

Hake defines \gls{ie} methods as those designed to promote conceptual understanding through ``heads-on'' and ``hands-on'' activities that yield immediate feedback through discussion. Conversely, traditional courses make little to no use of \gls{ie} methods, relying primarily on passive lectures, ``cookbook'' labs, and ``plug and chug'' problems.

\subsection{Examples of Interactive Engagement Methods}

\subsubsection{Just in Time Teaching}

\index{Just-in-Time Teaching}Just-in-Time Teaching (JiTT) is an \gls{ie} method that is based on creating an ongoing conversation between the students learning the material and professors teaching the material. Professors assign readings and practice problems that students are required to answer electronically before class starts. Then, the professor reads these responses and customizes the material covered in class that day\cite{novak1999}.

The ``just in time'' strategy has seen success not only in the classroom, but also in the business world. By constantly assessing the state of the class, it allows for focused lessons that cover the topics that students are having the most trouble with. Additionally, students become more interested in the course material when they have an active hand in shaping the structure of the lessons.

\subsubsection{Activity Based Physics and Workshop Physics}

Activity Based Physics is a multi-institutional project to sustain and enhance current efforts to render introductory physics courses more effective and exciting at both the secondary and college level.

This program represents a collaborative effort by an informally constituted team of educational reformers to use the outcomes of physics education research along with flexible computer tools to develop activity-based models of physics instruction.

This project includes the refinement of existing written materials, apparatus, instructional techniques, and computer software and hardware; the creation of new instructional materials and approaches; as well as classroom testing in different settings and dissemination.

The refinement and development of instructional strategies and materials is informed by a comprehensive program of educational research.

Workshop Physics is a new method of teaching calculus-based introductory physics without formal lectures. Instead students learn collaboratively through activities and observations. Observations are enhanced with computer tools for the collection, graphical display, analysis and modeling of real data. Workshop Physics classes consist of three, two-hour long sessions each week.

\subsubsection{Socratic Dialogue Inducing Methods}

Socratic Dialogue Inducing (SDI) methods teach students physics through hands-on experiements and discussion. These methods are designed to build student's mental model of a physics phenomenon by having students construct mental models of how they think the world works, testing these models with appropriate experiments, and then discussing their results in order to refine the model. Throughout the semester, the experiments become more and more sophisticated, refining their mental models of physics phenomena. Hake describes these labs as ``guided construction'' of physics concepts rather than ``guided discovery''\cite{hake1992}.

\subsection{Advantages of Interactive Engagement Methods}

\gls{ie} methods have been shown to maximize the efficacy of the classroom session, where human instructors are present and to structure the out-of-class time for maximum learning benefit. Hake's study of \gls{ie} classrooms across the United States has shown that students do, in fact, learn more from hands-on engagement than from any sort of passive learning.

Students in \gls{ie} classrooms are more engaged than those in traditional classrooms becasue they feel like their opinions actually matter. Their questions get a substantial amount of discussion time in class, so they feel like it is worth it to speak up. Also, students that work with their colleagues and professors are learning not only the material in class, but also how to work on a team. Additionally, this translates to a maximal retention of knowledge for the most students\cite{novak1999}.

\subsection{Disadvantages of Interactive Engagement Methods}

Although \gls{ie} methods are a boon for the physics student, they do not come without their challenges. It takes time and energy to create a good \gls{ie} curriculum. So much of the curriculum is based on reacting to student quetions in class. Thus, teachers need to be prepared to ``go off topic'' in an elegant and efficient way. This often means more preparation before class and more time spent on individual topics in class. Thus, a class that truly wants to use \gls{ie} methods cannot cover as much material of the course of the semester because there needs to be time for discussion.

Students who are used to traditional teaching methods might feel overwhelmed by \gls{ie} methods since they are so foreign - a situation called expectation violation. Coping with expectation violation is a normal part of life, but it can become a problem when the violation becomes very large. There are many ways to overcome expectations violation. First, students need to have a very clear understanding of what is expected of them. They need to know what should be turned in for credit, and what is only there for discussion and further learning. Additionally, students need to understand the philosophy behind the teaching style. For example, instructors need to tell students that going through a problem in multiple ways can help them tie concepts together that traditional methods would not be able to do.

One of the main advantages of passive lecturing is that anybody can do it. As long as the professor has knowledge of the subject area, it is pretty easy to throw together a lesson in a short amount of time (whether or not that lesson is good is another story entirely). Conversely, \gls{ie} methods take time and energy to implement. Being a skilled teacher involves collecting student feedback and changing how your lesson is structured accordingly. Professors need to take the time to effectively implement \gls{ie} strategies in the classroom. If a classroom has used traditional teaching techniques for a long time, the department might have to retrain many of the faculty to use \gls{ie} methods.

One of the key aspects of \gls{ie} methods is ongoing, personal feedback. This feedback can come in many forms - actual conversations with teachers and colleagues, online communication through email, interactive textbooks, or helpful feedback on assignments and grades. Distance learning students might be at a disadvantage with \gls{ie} methods because they physically cannot participate in the conversation that is occuring in a classroom. These students must also deal with many other problems like time zone differences or demanding work schedules. There are some solutions to this problem; programs like Adobe Connect, Cisco WebEx, and Skype make it easy for students to connect with the class as long as they have an internet connection. \gls{ie} methods like \gls{jitt} use asynchronous communciation methods that take advantage of these online tools. However, it is easy for a student to lose pace with the class when he or she is not physically present.
