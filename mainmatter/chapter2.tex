\chapter[Literature Review]{Literature Review}

\section{Background}

A large amount of work has gone into the study of how students learn new concepts. In this section I will first go over the learning philosophy of constructivism. I will then describe how this philosophy relates to problem solving. Next, I will describe how scaffolding methods can help enhance student problem solving skills. Finally, I will finish the discussion with how \gls{ie} methods have shaped physics education research in recent years.

\section{Constructivism}

The learning philosophy of constructivism When learning new material students bring with them the prior knowledge that they had of the subject at hand. This knowledge includes both correct models of how the world works as the research teamll as incorrect, pre-existing notions of the universe. These pre-existing notions might diverge from what the teacher is trying to teach.

Our model of learning in this study is constructivism. Constructivism is a learning philosophy grounded on the theory that the research team construct a personal understanding of the world the research team live in by reflecting on our experiences. We each generate our own mental models to make sense of our experiences. Learning is the process of adjusting our mental models with each new experience.

In the realm of physics, the research team create mental models of how the world works as the research team grow up. Some of these models are generally correct (i.e. if the floor is slippery, I might have trouble walking since I will slip). Other models are flathe research teamd (i.e. gravity always pulls things “downwards”). We will use a scaffolding approach to teach introductory physics to college level students.

It is tempting to quantify how ingrained a model is in student's heads. Zeilik and Bisard call deeply ingrained misconceptions that are hard to change structural misconceptions. Alternatively, they call easily changed misconceptions factual misconceptions\cite{zeilik2000}. However, a pre- and post concept inventory shothe research teamd no difference in gains betthe research teamen these two distinctions. However, even though the data shothe research teamd no difference, they still found it useful to track misconceptions in order to develop lessons for their courses.

\section{Problem Solving}

\subsection{Polya's Strategy for Problem Solving}

The modern theory of problem solving is rooted in the work of George Polya\cite{polya1985}. He outlined a deceptively simple strategy for solving a problem:

\begin{enumerate}
\item Understand the Problem
\item Devise a Plan
\item Execute the Plan
\end{enumerate}

Although this seems trivial, most problem-solvers do not make it past the first step!

\subsubsection{Understand the Problem}

Understanding the problem is more than just knowing the variables that one is given. One must take the time to fully understand the domain of what is being asked. For example, suppose a physics student is given a word problem. He or she will probably correctly identify what they need to solve for and what they are given. However, there are many aspects of the problem that go un-analyzed. Is there enough information to solve the problem? What assumptions do they need to make in order to solve the problem? Can they restate the problem in a simpler way? What is the required tolerance for their answer? All of these details are important to obtain a correct solution to the problem.

\subsubsection{Devise a Plan}

There are a huge number of strategies that can be used to solve a problem. On one extreme, a student might use the guess and check strategy to try to find the answer. On the other extreme, a student might work backwards from a known model to try to identify similarities between what they know and what they need to find. The student from above needs to take the time to figure out what the best strategy to solving the problem might be. More often than not, if a student does not have a well-defined plan to solve a problem, he or she will turn to the textbook to provide a plan for them. This textbook based plan does not always apply to the problem at hand.

\subsubsection{Execute the Plan}

Executing the plan does not just mean plugging numbers into equations. It is a process by which one checks his or her work to make sure that it is in line with their understanding of the problem and the plan that he or she devised. For the student above, he or she needs to carefully go through the solution, checking his or her work for math or conceptual errors. Additionally, the final step to the problem solving process should be checking the answer to see if it makes sense.

\subsection{Problem Solving in Physics}

There have been many studies that investigate how novice students solve problems in physics. Introductory students tend to solve problems by memorization and surface features. This might include identifying two different problems as being similar just because the diagrams are similar, looking through the textbook for the ``right equation'', or looking for similar solutions online\cite{maloney1994, walsh2007}. Studies have shown that novices often work backwards in problem solving - skipping the initial analysis described by Polya above and working towards finding an equation that fits the given variables.

On the other hand, expert problem solvers follow a method similar to that described by Polya above. They start by characterizing the qualitative aspects before developing a mathematical solution. They are not distracted by similar looking diagrams or solutions stated in the book. They start from first principles and work their way towards an answer\cite{larkin1980}.

\section{Scaffolding}

Students in a classroom need to contend with a variety of outside influences. For example, they need to figure out what specifically is important to learn and how to learn it in the most efficient way possible. Additionally, \gls{stem} students need to be able to keep up with changing themes in their field, analyze this new information, and use it to solve complex problems. Ultimately, professors would like their students to become independent learners that will continue to study on their own with limited support. Scaffolding provides an environment where students can slowly learn concepts, all the while gaining independence\cite{larkin2002}.

The method of scaffolding is based on the work of Vygotsky. He proposed that children could accomplish sophisticated tasks that were normally outside of their reach with the assistance of an adult. The adult creates a structure of small steps that the child can easily follow. Then, the child can connect each step to arrive at the final answer. This mindset has been extended to the classroom in general. Scaffolding refers to a variety of techniques that are used to slowly move a student towards a firm understanding of the material. Teachers provide successive levels of temporary support that students can use to structure their learning. These levels of support are removed when the student no longer needs them.

In our case the scaffolding will come from the interactive examples. Our scaffolding is unique in that students will have multiple paths to follow on their tutorials, thus making the scaffolding more immersive. Students will have a chance to reflect on the examples and talk amongst each other through Piazza, keeping with the spirit of \gls{ie}.

\section{Interactive Engagement Methods}

\gls{ie} methods were created to tackle the challenges of diverse classrooms. Students respond to a given lesson in a myriad of ways depending on their circumstances; today's professors have to deal with ``non-traditional students'' with a variety of backgrounds including older age students, full-time workers, part-time workers, commuters, military cadets, and special-needs students. Additionally, students have different educational backgrounds, career interests, and life goals that cannot be covered by a ``one size fits all'' approach\cite{choy2002, horn1996, novak1999}.

\gls{ie} methods challenge students to think about concepts in a deeper, more meaningful way. They frown upon generic problems that can be solved in one or two steps - so called ``plug and chug'' problems. Instead, \gls{ie} methods encourage students to frequently interact with their instructor, their teaching assistants, and their colleagues in order to solve deep, meaningful problems that draw together many areas of physics and mathematics. These problems often have real-world context that draw students interest.

Hake defines \gls{ie} methods as those designed to promote conceptual understanding through ``heads-on'' and ``hands-on'' activities that yield immediate feedback through discussion. Conversely, traditional courses make little to no use of \gls{ie} methods, relying primarily on passive lectures, ``cookbook'' labs, and ``plug and chug'' problems\cite{hake1998}. In his study of six thousand introductory physics students, he showed that classrooms that employed \gls{ie} methods show a definite increase in conceptual understanding over those that only employ traditional methods.

\subsection{Examples of Interactive Engagement Methods}

\subsubsection{Just in Time Teaching}

\index{Just-in-Time Teaching}\gls{jitt} is an \gls{ie} method that is based on creating an ongoing conversation between the students learning the material and professors teaching the material. Professors assign readings and practice problems that students are required to answer electronically before class starts. Then, the professor reads these responses and customizes the material covered in class that day\cite{novak1999}.

The ``just in time'' strategy has seen success not only in the classroom, but also in the real world. By constantly assessing the state of the classroom, it allows for focused lessons that cover the topics that students are having the most trouble with. Additionally, students become more interested in the course material when they have an active hand in shaping the structure of the lessons\cite{novak1999}.

\subsubsection{Activity Based Physics and Workshop Physics}

Activity Based Physics is a multi-institutional project that is trying to implement current \gls{ie} methods at the secondary school and college levels. This collaborative effort strives to create a series of online tools that can be used to model physics concepts. Additionally, the Activity Based Physics project develops new instructional materials and approaches that are dissemminated across the country.

\subsubsection{Workshop Physics}

Workshop Physics is an \gls{ie} method of teaching calculus-based introductory physics without any sort of lecture. Instead of formal lectures, students learn collaboratively through activities, experiments, and observations. Traditionally, Workshop Physics classes consist of three integrated lab and discussion sessions each week. However, these sessions can be adjusted to fit the needs of each individual classroom.

\subsubsection{Socratic Dialogue Inducing Methods}

Socratic Dialogue Inducing (SDI) methods teach students physics through hands-on experiements and discussion. These methods are designed to build student's mental model of a physics phenomenon by having students construct mental models of how they think the world works, testing these models with appropriate experiments, and then discussing their results in order to refine the model. Throughout the semester, the experiments become more and more sophisticated, refining their mental models of physics phenomena. Hake describes these labs as ``guided construction'' of physics concepts rather than ``guided discovery''\cite{hake1992}.

\subsubsection{Peer Instruction}

Eric Mazur developed the method of Peer Instruction in 1991. This approach involves pairing group work and individual work within an \gls{ie} classroom. Students first attempt homework problems by themselves. Then, they collaborate with each other, comparing their answers and analyzing where they might have gone wrong. This discussion enhances student learning\cite{mazur1997}.

\subsection{Advantages of Interactive Engagement Methods}

\gls{ie} methods have been shown to maximize the efficacy of the classroom session, where human instructors are present and to structure the out-of-class time for maximum learning benefit. Hake's study of \gls{ie} classrooms across the United States has shown that students do, in fact, learn more from hands-on engagement than from any sort of passive learning\cite{hake1998}.

Students in \gls{ie} classrooms are more engaged than those in traditional classrooms becasue they feel like their opinions actually matter. Their questions get a substantial amount of discussion time in class, so they feel like it is worth it to speak up. Also, students that work with their colleagues and professors are learning not only the material in class, but also how to work on a team. Additionally, this translates to a maximal retention of knowledge for the most students\cite{novak1999}.

\subsection{Disadvantages of Interactive Engagement Methods}

Although \gls{ie} methods are a boon for the physics student, they do not come without their challenges. It takes time and energy to create a good \gls{ie} curriculum. So much of the curriculum is based on reacting to student questions in class. Thus, teachers need to be prepared to ``go off topic'' in an elegant and efficient way. This often means more preparation before class and more time spent on individual topics in class. Thus, a class that truly wants to use \gls{ie} methods cannot cover as much material over the course of the semester because there needs to be time for discussion.

Students who are used to traditional teaching methods might feel overwhelmed by \gls{ie} methods since they are so foreign - a situation called expectation violation. Coping with expectation violation is a normal part of life, but it can become a problem when the violation becomes very large\cite{gigliotti1987, marian2006}. There are many ways to overcome expectations violation. First, students need to have a very clear understanding of what is expected of them. They need to know what should be turned in for credit, and what is only there for discussion and further learning. Additionally, students need to understand the philosophy behind the teaching style. For example, instructors need to tell students that going through a problem in multiple ways can help them tie concepts together that traditional methods would not be able to do.

One of the main advantages of passive lecturing is that anybody can do it. As long as the professor has knowledge of the subject area, it is pretty easy to throw together a lesson in a short amount of time (whether or not that lesson is good is another story entirely). Conversely, \gls{ie} methods take time and energy to implement. Being a skilled teacher involves collecting student feedback and changing how your lesson is structured accordingly. Professors need to take the time to effectively implement \gls{ie} strategies in the classroom. If a classroom has used traditional teaching techniques for a long time, the department might have to retrain many of the faculty to use \gls{ie} methods.

One of the key aspects of \gls{ie} methods is ongoing, personal feedback. This feedback can come in many forms - actual conversations with teachers and colleagues, online communication through email, interactive textbooks, or helpful feedback on assignments and grades. Distance learning students might be at a disadvantage with \gls{ie} methods because they physically cannot participate in the conversation that is occuring in a classroom. These students must also deal with many other problems like time zone differences or demanding work schedules. There are some solutions to this problem; programs like Adobe Connect, Cisco WebEx, and Skype make it easy for students to connect with the class as long as they have an internet connection. \gls{ie} methods like \gls{jitt} use asynchronous communciation methods that take advantage of these online tools. However, it is easy for a student to lose pace with the class when he or she is not physically present.
