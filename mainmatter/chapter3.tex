\chapter[Chapter 3: Interactive Engagement Methods]{Interactive Engagement Methods}

\section{Background}

Interactive engagement (IE) methods have been created to tackle many of the challenges in diverse classrooms. Students are no longer homogenous entities that all respond to a given lesson in a predictable way; professors have to deal with \"non-traditional students\" with a variety of backgrounds including older age students, full-time workers, part-time workers, commuters, military cadets, and special-needs students. Additionally, students have different educational backgrounds, career interests, and life goals that require individualized instruction\cite{novak1999}.

\section{Examples}

\subsection{Just in Time Teaching}

Just-in-Time Teaching (JiTT) is based on creating a conversation between the students learning the material and professors teaching the material. Professors assign readings and practice problems that students are required to answer electronically before class starts. Then, the professor reads these responses and customizes the material covered in class that day\cite{novak1999}.

The \"just in time\" strategy has seen success not only in the classroom, but also in the business world. By constantly assessing the state of the class, it allows for focused lessons that cover the topics that students are having the most trouble with.

\subsection{Workshop Physics}
\subsection{Tutorial-Based Instruction}
\subsection{Socratic Dialogue Methods}
\subsection{Active Learning Problem Sets}
\subsection{Context-Rich Problem Based Learning}

\section{Advantages of IE Methods}

\subsection{Classroom Enjoyment}

\subsection{Retention}

IE methods have been shown to
    1. To maximize the efficacy of the classroom session, where human instructors are present.
    2. To structure the out-of-class time for maximum learning benefit.

\subsection{Fostering Teamwork}

Students that work with their colleagues and professors are learning not only the material in class, but also how to work on a team. This translates to a maximal retention of knowledge for the most students\cite{novak1999}.

\section{Disadvantages of IE Methods}

\subsection{Time}
\subsection{Cost}
\subsection{Quantity of Material}

\subsection{Training}

IE methods are not something that you can simply wake up and start doing. Even skilled teachers need time to learn how to effectively implement these strategies in the classroom.

\subsection{Needs Facetime}

Distance learning students might be at a disadvantage with IE methods because they physically cannot participate in the conversation that is occuring in class.