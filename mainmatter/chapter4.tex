\chapter[Chapter 4: Evaluating Classroom Performance]{Evaluating Classroom Performance}

\section{Background}

\section{Exams and Quizzes}

\section{Homework}

\section{Concept Assessment Instruments}

Concept inventories and assessment instruments are a useful method of assessing student knowledge of the material; however, they are not simply tests that can be quickly put together and administered year after year. Lindell and Ding describe how it takes years to determine the validity and reliability of the results of a given concept inventory. They describe how reliability (precision) and validity (accuracy) play a role in the inventories and cite how factors such as age, course structure, geography, language, delivery of the tests, and wording of questions can influence the results of the assessment\cite{lindell2012}.

\subsection{Force Concept Inventory}

The force concept inventory (􏰀FCI)􏰁 is a multiple choice test that is used to measure a student's understanding of introductory mechanics. It is given at the beginning of an introductory mechanics course as a pre-test and again at the end of the course as a post-test. The pool of answers on the test are designed to correspond to common student misconceptions of mechanics; they were developed through a series of student interviews\cite{hestenes1992}.

Coletta et al. have observed a strong correlation between the normalized gain on the FCI and SAT scores. They go so far as to state that SAT scores might be a good indicator of the expected normalized gains within a classroom\cite{coletta2007}.

\section{Other Methods}

\section{Numerical Measures}

\subsection{Hake Factor (Gain Index)}

Normalized gain (G) vs. normalized gain <g>

\subsection{Gender Gap}