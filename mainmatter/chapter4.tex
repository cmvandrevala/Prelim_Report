\chapter[Chapter 4: Current Progress]{Current Progress}

\section{Background}

Work started on the CITA on CHIP project in earnest in February 2015.

Personalization is important
Analytics and tracking progress
It might be able to better serve women and minorities in STEM
Focus on student’s questions (it promotes individual thinking)
Track progress in an individualized way
Coach students through problems
Consider multiple iterations of the same lesson
Stress problem solving strategies
This program could be an equalizer in the classroom
This program might boost student confidence
This program might be good for international students with poor English
Stress multiple learning mechanisms
Break up each problem into tiny pieces
Provide immediate feedback
This program may help customize the curriculum
It is a non-linear way to solve problems

\section{Analysis Tools}

\subsection{CHIP}

The CHIP program has a set of built in tools.

\subsection{R}

\subsection{Muffin}

Over the course of this project, we will have to deal with data from a variety of sources. For example, we will have to review CSV data from the CHIP website, JSON data from the Piazza website, audio data from focus group transcripts, and written data from notes taken by teaching assistants over the course of the study. We need a simple and efficient way to translate one data format into another so that it can be used with \gls{chip}.

I have written an open-source data analysis toolbox called Muffin that will handle this information. Muffin is a set of Ruby functions that will simply read and write data to a format appropriate for our study. The source code for Muffin is hosted on \index{RubyGems}RubyGems as well as on my personal \index{GitHub}GitHub page.

Muffin is simply a toolbox that converts data from one format to another. It does not store anything or conduct analysis in any way. Thus, privacy is maintained, even when the Muffin source code is freely distributed online.

\section{Development of CITA}

We only continue exploring a topic if we feel comfortable in our environment and we feel it is worth it. We are economic creatures.

Goals
With the increased popularity of the online course as well as increased enrollment in the course as a whole, students need a homework website with an interactive learning tool that guides them through the homework no matter what time of day (i.e. different time zones, work schedules).  We want to design an computerized interactive TA (iTA) that is available at any time.

We want to design interactive components for each individual problem that can help students develop an approach to problem solving.  The CITA will guide students through problems in such a way as to develop critical thinking and problem solving skills.

High school teachers want to take the course to become accredited.  They see how physics should be thought out / structured.

High Priority Goals
- Update all content with \gls{cita} hints
- Update the statistical analysis parts of the site for student learning outcomes (interactive graphs of statistics)
- Review the site and bring all useful components to the forefront

Medium Priority Goals
- Update site graphics
-> Faculty First
-> Students Second
- Update figures (interactive?)

Low Priority Goals
- Blackboard/Canvas/Whatever Integration
- Mobile Support?

\subsection{Shallow CITA}

\index{Shallow CITA}

\subsubsection{Postscripts}

\index{Postscripts}

\subsubsection{Estimation Problems}

\subsection{Immersive CITA}
\subsubsection{None}
\subsubsection{Linear}
\subsubsection{Branching}

\section{What CITA Does Accomplish}

\section{What CITA Does Not Accomplish}

It is important to set boundaries.

\subsection{It Does Not Replace Teachers}

CITA in no way is designed to replace the job of a teacher in the classroom.

\subsection{It is Not a Course Management Tool}



\section{Background}

\section{Building the System}

\subsection{Shallow CITA}

\index{Shallow CITA}Shallow CITA is used to catch all of the ``low-hanging fruit''.

Units
Question Phrasing
Some negative signs (some require more explanation)

\subsection{Immersive CITA}

\subsection{Postscripts}

\section{Research Question and Hypothesis}

\section{Methodology}

\section{Results}

\subsection{Summer 2014 vs. Summer 2015 Homework Scores}

INCLUDE BAR CHART COMPARING SCORES HERE. MAKE SURE THAT THE STANDARD DEVIATIONS ARE INCLUDED. BOX AND WHISKER PLOTS MIGHT ALSO BE APPROPRIATE.