\chapter[Coding Scheme]{Coding Scheme}

\section{Background}

A coding scheme is a standardized way of characterizing particpant answers in a qualitative study so that the researcher can identify common trends. In my study I am interested in identifying student reactions to the \gls{cita} system\cite{patton2015}. Thus, I am interested in identifying general information about the student and connecting it to his or her specific comments.

\section{Coding Scheme}

I will refrain from using the identifiers 24100 and 24100D because they may or may not correspond with online or on-campus student. For example, if a student enrolled in a certain section at the beginning of the semester, but then switched to a different class, the identifiers may not be entirely correct.

\subsection{Focus Groups, Interviews, CHIP, and Piazza}

In order to protect the identity of each student, we will disassociate their comments from their identity. Each anonymous comment will be identified using the following information.

\begin{enumerate}
\item Data Source (e.g. focus, interview, chip, or piazza)
\item Semester (e.g. fall2015, spring2016, summer2017)
\item Type of Instruction (e.g. online, on-campus)
\item Pseudonym (begins with the same letter as the student's name)
\end{enumerate}

For example, suppose a student named John Smith took the online section of PHYS 24100D in the summer of 2015. He kindly agreed to participant in a focus group discussion. He might be identified by ``focus\_summer2015\_online\_Jack''.

Suppose another student named Rebecca Jones took the on-campus section of PHYS 24100 in the spring of 2014. She agreed to participant in a focus group discussion and follow up with a personal interview. Then, she would be identified in two ways in the two studies: ``focus\_spring2014\_on-campus\_Rachel'' and ``interview\_spring2014\_on-campus\_Rhonda''.

\subsection{Course Evaluations}

I am also going to use the end-of-the-semester evaluations in order to identify any comments made about the homework system. These comments are completely anonymous - thus, there is no way that I can identify any sort of demographic data or connect the comment to student performance in the class. Each anonymous comment will be identified using the following information.

\begin{enumerate}
\item Data Source (e.g. evaluation)
\item Semester (e.g. fall2015, spring2016, summer2017)
\item Type of Instruction (e.g. online, on-campus, unknown)
\item Tracking Number (a number that identifies the comments in the order that they appear on the course evaluation)
\end{enumerate}

For example, suppose an anonymous student comments on how much he or she enjoyed using the \gls{cita} system. This happens to be the third comment on the list of comments received. We identify that he or she was in the online section of PHYS 24100D in the fall of 2016. I would identify the student as ``evaluation\_fall2016\_online\_3''.