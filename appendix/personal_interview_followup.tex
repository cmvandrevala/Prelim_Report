\chapter[Personal Interview Followup]{Personal Interview Followup}

\section{Background}

After a focus group session, participants will have the opportunity to contact any of the moderators to schedule a follow-up one-on-one interview. This interview is less structured than the focus group itself; it can be conducted over the phone, over email, or in person, depending on the desire of the participant. Ideally, we would like to interview the participant in person and record the meeting, just like with the focus group. There will be no extra reward for attending this followup interview.

This discussion guide is merely a set of seed questions that are used to spur conversation. If the participant wants to discuss something else, the moderator is encouraged to ``go off topic'' and really delve into what the participant has to say.

Notice that many of the discussion questions remain the same as in the focus group. However, the focus of these questions is no longer on broad information gathering; rather, it is specific data collection for a student.

\section{Opening Remarks}

Thank you for coming today. My name is [MODERATOR], and I am very glad that you followed up with us. I know that it can be difficult to voice all of your opinions in a focus group, so this is the perfect time to really dive into your personal thoughts and experiences with the homework system.

I would like to record the discussion so that we don’t miss anything that is said. Of course, this discussion will stay confidential and your comments will not affect your course grade in any way. Is it alright if I record our discussion?

I have allotted one and a half hours for our interview today, but can certainly schedule more time if it is needed. Do you have any questions before we start?


\section{Seed Questions}

\begin{enumerate}
	\item What is your area of study here at Purdue?
	\begin{itemize}
		\item Engineering vs. Science vs. Non-Traditional Major
	\end{itemize}
	\item What was your experience with previous physics assignments?
	\begin{itemize}
		\item Specific Online Homework Systems (Webassign, \gls{chip}, etc.)
		\item In-Class Written Assignments
		\item Systems Paired with a Textbook
	\end{itemize}
	\item What is your specific work flow with physics homework?
	\begin{itemize}
		\item Times
		\item Thoughts
		\item Study Groups
		\item Finding Answers on the Internet
	\end{itemize}
	\item How did the look of the website affect your learning?
	\begin{itemize}
		\item Graphics
		\item Diagrams
		\item Pictures
		\item Boxes
		\item Colors
	\end{itemize}
	\item What did you think of the flow of the website?
	\begin{itemize}
		\item Help Buttons
		\item Branching
		\item Navigation
		\item Returning to Earlier Sections
	\end{itemize}
	\item Did the content help you on specific homework problems?
	\begin{itemize}
		\item Relevancy of Tutorial
		\item Expectations of Assignment
		\item Giving Away Answers
		\item Step-by-Step Instructions
	\end{itemize}
	\item Did the content help you on the quizzes and exams?
	\begin{itemize}
		\item Similarity of Tutorials to Quizzes and Exams
		\item Seeing Parallels Between Homework and Exams
		\item Being Able to Answer a Question in Multiple Ways
	\end{itemize}
	\item Overall, do you feel that the interactive examples were helpful?
	\begin{itemize}
		\item Specific Problems Facing the Student
	\end{itemize}
	\item Do you have any final ideas about how we can improve the system?
	\begin{itemize}
		\item Learning Goals of Students
	\end{itemize}
\end{enumerate}