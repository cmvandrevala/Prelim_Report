\chapter[Multi-Step Problem]{Multi-Step Problem}

\section{Background}

In order to judge the efficacy of the interactive tutorials in \gls{cita}, we need to analyze how students work through a non-trivial problem. We are going to analyze student progress in a multi-step problem similar to the technique used by Bruce Sherwood\cite{sherwood2005}. However, we will expand on his numerical analysis by adding the concept of ``tags'' to our review. By comparing student data in classes with and without \gls{cita}, we will be able to gauge how well our scaffolding worked.

The sections below outline the problem that will be used, a suggested rubric for analyzing the student solutions, and a table of potential ``tags''. Of course, these tools might change with the changing requirements of the course.

\section{Problem Text}

A laser beam of power P = 10.0 W and diameter D = 1.00 mm is directed upward onto one circular face of a perfectly reflecting cylinder. The cylinder levitates due to the balance between the upward radiation force and the downward gravitational force.

If the density of the cylinder is 1.25 g/cm$^3$, and the diameter of the circular face is 0.50 mm, what is the height H of the cylinder?

\section{Suggested Rubric}

It is important to grade these multi-step problems in a consistent way. Shown below is a suggested rubric for analyzing student performance on the question above. All students start with a score of zero. Then, the grader moves down the rubric row by row, determining to the best of his or her ability if the student has shown the work required for the next checkpoint. If the grader feels that the student has shown the required work, he or she will increment the student's score by one and move onto the next checkpoint in the next row. If the grader does not feel that the student has the required work, then the student's score is recorded.

A student is allowed to skip steps. For example, if the student did not specifically ``Identify Principles'' from the rubric but instead moved on to a ``Free Body Diagram'', his or her score will be at least a three. Note that we will keep track of analyses that skip steps using one of the tags (see the next section).

The ``Examples'' column gives some potential ways that the student could accomplish a given checkpoint. A student need not accomplish all of the suggestions given in the ``Examples'' column - one suggestion is enough. Note that this is not an all-encompassing list; the grader must make a judgement call to determine whether or not the student earned the checkpoint.

\begin{table}[!ht]
  \centering
  \begin{tabular}{|l|l|l|}
    \hline
    \textbf{Score} & \textbf{Checkpoint} & \textbf{Examples}\\
	\hline
	0 & No Attempt & Wrote down incorrect, random, or irrelevant equations\\
	& & Gave no explanation or context for incorrect work\\
	& & Turned in a blank sheet of paper\\
	\hline
	1 & Start the Problem & Write out one or two correct equations\\
	& & Give some explanation or context of a potential solution\\
	& & Draw a diagram of the cylinder (with no force vectors)\\
	& & Draw a diagram of the laser beam (with no associated vectors)\\
	\hline
	2 & Identify Principles & Identify Newton's laws\\
	& & Identify radiation pressure\\
	& & Identify Poynting vector, energy, or momentum\\
	\hline
	3 & Free Body Diagram & Draw a correct free body diagram of the cylinder\\
	& & Expand on an original diagram, commenting on force or pressure\\
	\hline
	4 & Newton's Second Law & Equate gravitational force and radiation force\\
	& & Simplify the expression\\
	\hline
	5 & Solve for Mass & Get an analytic expression for the mass of the cylinder\\
	& & Get a numeric expression for the mass of the cylinder\\
	\hline
	6 & Solve for the Height & Get an analytic expression for the height of the cylinder\\
	& & Get a numeric expression for the height of the cylinder\\
	\hline
  \end{tabular}
  \caption{Suggested Rubric}
  \label{tab:rubric}
\end{table}

\section{Solution Tags}

After grading each problem using the rubric above, the grader will take a step back and look at the student's solution as a whole. If nothing in particular stands out to the grader, there is no need for further analysis. However, the grader can tag a student's homework with one of the following phrases. This is an additional way to determine the path that students followed as they solved the multi-step problem.

A tag does not affect the score in any way. It is just an identifier to expand on the analysis of the problem. A single solution can have multiple tags.

\pagebreak

\begin{landscape}
\begin{table}[!ht]
  \centering
  \begin{tabular}{|l|l|}
    \hline
    \textbf{Name} & \textbf{Description}\\
	\hline
	Skipped Steps & The student did not explicitly write out each of the steps in the rubric,\\
	& but still did well on the problem (a score of five or better).\\
	\hline
	Incorrect Algebra & The student would have received a score of six if it weren't for incorrect\\
	& algebra.\\
	\hline
	Random Solution & The student managed to score well on the problem (four or higher), but\\
	& only because he or she wrote out random equations in the hope that one or\\
	& more would earn them credit.\\
	\hline
  \end{tabular}
  \caption{Solution Tags}
  \label{tab:solution-tags}
\end{table}
\end{landscape}


