\chapter[Chapter 2: Theories of Learning]{Theories of Learning}

\section{Background}

\section{Paradigms of Learning}

\subsection{Behaviorist Theories}
\subsection{Cognitivist Theories}
\subsection{Constructivist, Social, and Situational Theories}
\subsection{Motivational and Humanist Theories}
\subsection{Design Theories and Models}
\subsection{Descriptive and Meta Theories}
\subsection{Identity Theotries}
\subsection{Miscellaneous Learning Theories and Models}

\section{My Focused Learning Theory}


http://www.learning-theories.com/

When learning new material students bring with them the prior knowledge that they had of the subject at hand. This knowledge includes both correct models of how the world works as well as incorrect, pre-existing notions of the universe. These pre-existing notions diverge from what the teacher is trying to teach.

Teaching learning gap (Redish and Steinberg 1999)

It is tempting to quantify how ingrained a model is in student's heads. Zeilik and Bisard call deeply ingrained misconceptions that are hard to change structural misconceptions. Alternatively, they call easily changed misconceptions factual misconceptions\cite{zeilik2000}. However, a pre- and post concept inventory showed no difference in gains between these two distinctions. However, even though the data showed no difference, they still found it useful to track misconceptions in order to develop lessons for their courses.

Zeilik et al 1997

I use the advice of Zeilik and  Bisard to track misconceptions\cite{zeilik2000}
If I want to teach as efficiently as possible, I need to target misconceptions.